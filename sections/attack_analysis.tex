\section{Attack Surface Analysis and Defense}

\subsection{Attack Classification}

Table~\ref{tab:attack_matrix} summarizes the primary vectors and their corresponding architectural defenses.

\begin{table}[htbp]
\centering
\caption{Attack Classification and Defense Mechanisms}
\label{tab:attack_matrix}
\small
\begin{tabularx}{\linewidth}{@{}l X X@{}}
\toprule
\textbf{Attack Vector} & \textbf{Description} & \textbf{Defense Mechanism} \\
\midrule
State Rollback & Broadcasting old $U_{\mathrm{state}}$ & \textbf{Monotonicity} + RefOp Binding \\
Topology Obf. & Rapid splicing to hide flow & DAA Fees + Value Conservation \\
PTLC Hijacking & Intercepting adaptors & \textbf{Onion Routing} + Blinded Locks \\
Resource Exh. & Recursive factory spam & State Rent + Merge Ops \\
Cross-Replay & Sig reuse across channels & \textbf{Domain Separation} \\
Pinning & Mempool congestion & \textbf{STPC Strategy} \\
\bottomrule
\end{tabularx}
\end{table}

\subsection{State Rollback Attack Analysis}

\subsubsection{Attack Vector}
An adversary broadcasts an outdated state $U_{\mathrm{state}}^{(n-k)}$ ($k > 0$) to revert balances.

\subsubsection{Defense Mechanisms}
\begin{enumerate}
    \item \textbf{Consensus Monotonicity}: Validators enforce $n_{\mathrm{new}} > n_{\mathrm{curr}}$.
    \item \textbf{RefOp Binding}: Signatures bind to the static anchor:
    $$ \sigma = \mathrm{Sign}_{sk}(\mathrm{state}_n \parallel \RefOp\text{-OutPoint}) $$
    Topology changes (Splice) alter the OutPoint, invalidating all prior signatures.
\end{enumerate}

\begin{theorem}[Rollback Resistance]
The probability of a successful rollback attack is bounded by:
$$ \Pr[\text{rollback}] \le \Pr[\text{51\% attack}] \times \Pr[\text{victim offline}] $$
Since both factors are small under normal network conditions, successful rollback is negligible.
\end{theorem}

\subsection{Topology Obfuscation}

\subsubsection{Mitigation}
To prevent malicious topology churn (e.g., for money laundering):
\begin{enumerate}
    \item \textbf{Value Conservation}: $\sum V_{\mathrm{in}} = \sum V_{\mathrm{out}} + \delta_{\mathrm{fee}}$.
    \item \textbf{DAA Timing Costs}:
    $$ \mathrm{Cost}_{\mathrm{obf}} = f_{\mathrm{splice}} \times \overline{\mathrm{Fee}}_{\mathrm{L1}} $$
    Rapid reconfiguration becomes prohibitively expensive on L1.
\end{enumerate}

\subsection{PTLC Hijacking}

\subsubsection{Defense Strategy}
\begin{enumerate}
    \item \textbf{Sphinx Onion Routing}:
    $$ M_i = \mathrm{Encrypt}(PK_i, \{\mathrm{next}, \mathrm{amt}, \mathrm{lock}\}) $$
    \item \textbf{Blinded Point Locks}:
    $$ Q_i = Q_{\mathrm{base}} + r_i \cdot G $$
    where $r_i$ is shared only between sender and receiver. Intermediate nodes cannot correlate $Q_i$ with $Q_{i+1}$.
\end{enumerate}

\subsection{Resource Exhaustion via Channel Proliferation}

Attackers may create deep recursive factories to bloat the UTXO set. A factory of depth $d$ with branching factor $k$ creates up to $k^d$ leaf channels.

\subsubsection{Economic Countermeasures: State Rent}
We introduce a depth-weighted rent function:
\begin{equation}
R_{\mathrm{total}} = R_{\mathrm{base}} \cdot (1 + \alpha \cdot d) \cdot \Delta t_{\mathrm{age}}
\end{equation}
where $d$ is the topology depth, $\alpha = 0.1$ is the depth penalty coefficient, and $\Delta t_{\mathrm{age}}$ is the channel age in DAA units. Unpaid rent can be claimed by any searcher via a \textbf{Merge Transaction}, incentivizing state pruning.

\subsection{Cross-Channel Replay}

\subsubsection{Domain Separation}
Signatures are bound to a unique channel context:
\begin{align*}
\mathrm{ChannelID} &= H(\mathrm{fund\_outpoint} \parallel \mathrm{nonce}) \\
\sigma &= \mathrm{Sign}_{sk}(H(\mathrm{domain} \parallel \mathrm{ChannelID}) \parallel m)
\end{align*}
Since \texttt{fund\_outpoint} is globally unique, cross-channel collisions are mathematically impossible ($P < 2^{-256}$).

\subsection{Pinning Attack Analysis}

\subsubsection{Mechanism Comparison}
In legacy LN, attackers use Child-Pays-For-Parent (CPFP) or RBF rules to ``pin'' a low-fee transaction in the mempool. STPC eliminates this.

\begin{figure}[htbp]
\centering
\begin{tikzpicture}[
    font=\sffamily\scriptsize,
    node distance=0.5cm and 0.5cm,
    box/.style={rectangle, draw, minimum width=1.6cm, minimum height=0.8cm, align=center, fill=white},
    mempool/.style={rectangle, draw, dashed, fill=gray!5, minimum width=3.2cm, minimum height=2.8cm, label={[anchor=north]north:Mempool Slot}},
    arrow/.style={->, >=Stealth, thick}
]

    % --- Left: Legacy RBF Pinning ---
    \begin{scope}[local bounding box=left]
        \node[font=\bfseries] at (0, 2.0) {Legacy: RBF Pinning};
        
        % Mempool Area
        \node[mempool] (pool1) at (0,0) {};
        
        % Inside Mempool
        \node[box, fill=red!10] (txA) at (0, 0.5) {Tx A\\(Low Fee)};
        \node[box, fill=red!20, minimum width=2.5cm] (child) at (0, -0.8) {Huge Child Tx\\(Pins A)};
        \draw[arrow] (child) -- (txA);
        
        % Outside Mempool (Honest Tx)
        \node[box, fill=blue!10, left=1.2cm of pool1.west, anchor=east] (txB) {Honest Tx B\\(Higher Fee)};
        
        % Blocked Arrow
        \draw[arrow, red] (txB.east) -- node[above, font=\tiny, text=red] {Blocked} node[below, font=\tiny, text=red] {Rule 3} (pool1.west);
        \node[text=red, font=\tiny] at ($(pool1.west) + (-0.2, 0)$) {$\times$};
    \end{scope}

    % --- Right: Eltoo 2.0 STPC ---
    \begin{scope}[xshift=6.5cm, local bounding box=right]
        \node[font=\bfseries] at (0, 2.0) {Eltoo 2.0: STPC};
        
        % Mempool Area
        \node[mempool] (pool2) at (0,0) {};
        
        % Inside Mempool (Attack)
        \node[box, fill=red!10] (stA) at (0, 0) {State $n$\\Attack};
        \node[above=0.1cm of stA, text=red, font=\tiny] {Evicted};
        \draw[->, red, dashed, bend right] (stA.east) to[out=-30, in=30] ($(pool2.east)+(0.2,0)$);

        % Outside Mempool (Honest Tx)
        \node[box, fill=green!10, left=1.2cm of pool2.west, anchor=east] (stB) {State $n+1$\\Honest};
        
        % Replaces Arrow
        \draw[arrow, green!60!black] (stB.east) -- node[above, font=\tiny] {Replaces} node[below, font=\tiny] {Monotonicity} (pool2.west);
    \end{scope}

\end{tikzpicture}
\caption{Pinning Attack Defense. Left: Legacy RBF rules allow attackers to pin transactions using heavy child descendants (Rule 3 blocking). Right: STPC enforces unconditional replacement based on state sequence ($n+1 > n$), ignoring descendant weight.}
\label{fig:pinning_defense}
\end{figure}

\begin{theorem}[Pinning Immunity]
Under STPC, the expected confirmation time for the highest-sequence state is bounded by:
$$ \mathbb{E}[T_{\mathrm{confirm}}] \le \frac{1}{\lambda_{\mathrm{block}}} \cdot (1 + \epsilon_{\mathrm{jitter}}) $$
where $\lambda_{\mathrm{block}}$ is the block arrival rate ($\sim$1 block/second for Kaspa) and $\epsilon_{\mathrm{jitter}} \approx 0.1$ accounts for network propagation variance.
\end{theorem}

\subsection{Griefing Attack Cost Analysis}

\begin{table}[htbp]
\centering
\caption{Griefing Cost Comparison}
\label{tab:griefing}
\small
\begin{tabularx}{\linewidth}{@{}l X X@{}}
\toprule
\textbf{Metric} & \textbf{Attacker Cost} & \textbf{Victim Cost} \\
\midrule
Spam States & $\mathcal{O}(N) \times \mathrm{Fee}$ & $\mathcal{O}(1)$ Verify \\
Force Close & $1 \times \mathrm{Fee}$ & $1 \times \mathrm{Fee}$ \\
Fund Lock & Capital Opportunity Cost & Capital Opportunity Cost \\
Time Cost & Days (Legacy) & \textbf{Seconds (Eltoo 2.0)} \\
\bottomrule
\end{tabularx}
\end{table}

\subsection{Security Summary}

\begin{table}[htbp]
\centering
\caption{Security Architecture Comparison}
\label{tab:sec_comparison}
\scriptsize
\begin{tabularx}{\linewidth}{@{}l l l X@{}}
\toprule
\textbf{Vector} & \textbf{Lightning} & \textbf{BIP-118} & \textbf{This Work} \\
\midrule
State Theft & High & Medium & \textbf{Atomic} \\
Replay & Medium & Medium & \textbf{Domain Sep.} \\
DoS Cost & Low & Medium & \textbf{High ($\times N$)} \\
Pinning & High & Medium & \textbf{Immune} \\
Offline & Hours & Days & \textbf{Weeks} \\
\bottomrule
\end{tabularx}
\end{table}

This architecture achieves superior security through consensus-layer enforcement and economic alignment, removing the game-theoretic fragility of penalty-based systems.
