\appendix

\section{Glossary and Preliminaries}
\label{appendix:glossary}

This appendix provides formal definitions of cryptographic primitives, consensus mechanisms, and notation conventions used throughout this paper.

\subsection{Ledger Model and Transaction Structure}

\begin{definition}[UTXO (Unspent Transaction Output)]
The ledger model used by Bitcoin and its derivatives. Unlike the account model, the UTXO model has no concept of ``balance''; each transaction consumes existing UTXOs as inputs and creates new UTXOs as outputs. Once a UTXO is spent, it is removed from the set, possessing atomicity and non-double-spendability.
\end{definition}

\begin{definition}[Transaction Malleability]
A vulnerability where a transaction's identifier (TxID) could be modified by a third party after signing. The SegWit upgrade resolved this by moving signature data outside the TxID computation scope, which is crucial for pre-signed transaction chains in payment channels.
\end{definition}

\subsection{Payment Channel Fundamentals}

\begin{definition}[Payment Channel]
An off-chain payment mechanism established between two or more parties, requiring on-chain transactions only for channel opening (Funding) and closing (Settlement), with intermediate state updates completed entirely off-chain.
\end{definition}

\begin{definition}[State Channel]
A generalization of payment channels supporting arbitrary state transitions rather than just payment balance updates.
\end{definition}

\begin{definition}[Channel Factory]
A shared on-chain funding pool created by multiple parties that can dynamically spawn multiple bilateral or multilateral sub-channels without requiring on-chain transactions for sub-channel opening and closing.
\end{definition}

\begin{definition}[Watchtower]
A proxy node that monitors on-chain activity on behalf of offline users and broadcasts penalty or update transactions to prevent counterparties from broadcasting stale states.
\end{definition}

\subsection{Conditional Payment Primitives}

\begin{definition}[HTLC (Hash Time-Locked Contract)]
A conditional payment primitive where the recipient must provide preimage $r$ such that $H(r) = h$ before the timelock expires to claim funds; otherwise, funds are refunded to the sender. HTLCs form the foundation of Lightning Network multi-hop payments.
\end{definition}

\begin{definition}[PTLC (Point Time-Locked Contract)]
A privacy-enhanced version of HTLC using elliptic curve point $R = r \cdot G$ instead of hash values. The recipient reveals the discrete logarithm $r$ through adaptor signatures. PTLCs eliminate cross-channel payment correlation.
\end{definition}

\subsection{Cryptographic Foundations}

\begin{definition}[Elliptic Curve Group]
The elliptic curve used in this paper is secp256k1, defined over the finite field $\mathbb{F}_p$. Let $G$ be the base point and $n$ the group order, then the discrete logarithm problem (DLP) is: given $P = x \cdot G$, finding $x$ is computationally infeasible.
\end{definition}

\begin{definition}[Schnorr Signature]
Schnorr signature is a digital signature scheme based on the discrete logarithm problem. Given elliptic curve group with generator $G$ and order $n$, private key $x \in \mathbb{Z}_n$, and public key $P = x \cdot G$, the signing process for message $m$ is:
\begin{enumerate}
    \item Choose random nonce $k \in \mathbb{Z}_n$, compute $R = k \cdot G$
    \item Compute challenge $e = H(R \| P \| m)$
    \item Compute $s = k + e \cdot x \mod n$
    \item Signature is $(R, s)$
\end{enumerate}
The \textbf{linearity property} of Schnorr signatures ($s_1 + s_2$ corresponds to $P_1 + P_2$) is the mathematical foundation for multi-signature aggregation (MuSig2) and adaptor signatures.
\end{definition}

\begin{definition}[MuSig2 Multi-Party Signature]
MuSig2 is an interactive multi-party signature protocol that allows $n$ participants to jointly generate a single aggregated signature. Let the set of participant public keys be $\{P_1, \ldots, P_n\}$, the aggregated public key is:
$$P_{agg} = \sum_{i=1}^{n} a_i \cdot P_i, \quad \text{where } a_i = H(L \| P_i), L = H(P_1 \| \cdots \| P_n)$$
MuSig2 reduces one round of interaction compared to the original MuSig, requiring only two rounds to complete signing.
\end{definition}

\begin{definition}[Adaptor Signature]
Adaptor signature is an ``incomplete'' pre-signature $\tilde{\sigma}$ that requires knowledge of a secret value $t$ to be converted into a valid signature $\sigma$:
$$\sigma = \text{Adapt}(\tilde{\sigma}, t)$$
Conversely, anyone observing $(\tilde{\sigma}, \sigma)$ can extract the secret value:
$$t = \text{Extract}(\tilde{\sigma}, \sigma)$$
Adaptor signatures achieve ``atomic revelation'': when one party claims funds, they necessarily reveal the secret value, which is the cryptographic basis for PTLCs and cross-chain atomic swaps.
\end{definition}

\begin{definition}[Hash Function and Commitment]
The hash function $H: \{0,1\}^* \to \{0,1\}^{256}$ used in this paper satisfies the following security properties:
\begin{itemize}
    \item \textbf{Preimage resistance}: Given $h$, finding $m$ such that $H(m) = h$ is computationally infeasible
    \item \textbf{Collision resistance}: Finding $m_1 \neq m_2$ such that $H(m_1) = H(m_2)$ is computationally infeasible
\end{itemize}
Hash commitment $c = H(m \| r)$ possesses hiding and binding properties, widely used in HTLCs and state commitments.
\end{definition}

\subsection{Timelock Mechanisms}

\begin{definition}[Timelock]
Timelock is a consensus mechanism that renders a transaction invalid before a specific time or block height. This paper involves two types of timelocks:

\begin{table}[htbp]
\centering
\begin{tabular}{@{}llll@{}}
\toprule
Type & Mechanism Name & Lock Basis & Application Scenario \\
\midrule
Absolute & nLocktime & Block height or Unix timestamp & HTLC timeout refund \\
Relative & CSV (BIP-112) & Blocks after UTXO confirmation & Channel dispute period \\
\bottomrule
\end{tabular}
\end{table}
\end{definition}

\begin{definition}[DAA Score]
In GhostDAG consensus, the Difficulty Adjustment Algorithm Score provides a globally monotonically increasing logical clock. Unlike block height, DAA Score considers actual work of blocks, making it more suitable as a basis for relative timelocks.
\end{definition}

\subsection{Directed Acyclic Graph Consensus}

\begin{definition}[GhostDAG Protocol]
Traditional blockchains adopt linear chain structures, producing ``orphan blocks'' under network delay. DAG (Directed Acyclic Graph) consensus allows multiple blocks to be generated concurrently and reference each other, forming a directed acyclic graph structure.

Core parameters of the GhostDAG protocol:
\begin{itemize}
    \item \textbf{$D$ (network delay bound)}: Maximum propagation delay between honest nodes
    \item \textbf{$k$ (blue set parameter)}: Determines protocol's security-liveness tradeoff
\end{itemize}

The protocol achieves total ordering by defining ``blue sets'' for blocks:
$$\forall b_1, b_2 \in \text{DAG}: b_1 \prec_{\text{blue}} b_2 \iff \text{BlueScore}(b_1) < \text{BlueScore}(b_2)$$
where $\text{BlueScore}(b)$ is computed recursively based on the block's position in the DAG and its relationship to the ``blue'' (honest) cluster.
\end{definition}

\subsection{Finite State Machine Foundations}

\begin{definition}[Finite State Machine]
A finite state machine (FSM) is a five-tuple $M = (Q, \Sigma, \delta, q_0, F)$:
\begin{itemize}
    \item $Q$: Finite set of states
    \item $\Sigma$: Input alphabet (set of events/inputs)
    \item $\delta: Q \times \Sigma \to Q$: State transition function
    \item $q_0 \in Q$: Initial state
    \item $F \subseteq Q$: Set of final states
\end{itemize}
\end{definition}

\begin{definition}[State Machine Determinism]
If for any state $q \in Q$ and input $\sigma \in \Sigma$, $\delta(q, \sigma)$ has at most one result, then $M$ is a deterministic finite automaton (DFA). The channel state machines in this paper strictly satisfy the determinism condition.
\end{definition}

\subsection{Covenants and Script Extensions}

\begin{definition}[Covenant]
A covenant is a mechanism that imposes constraints on how a UTXO can be spent in the future. Formally, a covenant is a predicate $C: \text{Tx} \to \{0, 1\}$, where spending transaction $\tau$ must satisfy $C(\tau) = 1$.

Covenant classification:
\begin{itemize}
    \item \textbf{Non-recursive covenants}: Constraints apply only to direct spending transactions, e.g., CLTV, CSV
    \item \textbf{Recursive covenants}: Constraints can propagate to subsequent transactions, e.g., CTV (BIP-119), APO (BIP-118)
\end{itemize}
\end{definition}

\begin{definition}[SIGHASH Flags]
SIGHASH flags determine which parts of a transaction are covered by a Schnorr/ECDSA signature:

\begin{table}[htbp]
\centering
\small
\begin{tabular}{@{}llll@{}}
\toprule
\textbf{Flag} & \textbf{Covers Inputs} & \textbf{Covers Outputs} & \textbf{Use Case} \\
\midrule
\texttt{SIGHASH\_ALL} & All & All & Standard transactions \\
\texttt{SIGHASH\_NONE} & All & None & Allow receiver to add outputs \\
\texttt{SIGHASH\_SINGLE} & All & Matching index & Multi-party tx construction \\
\texttt{SIGHASH\_ANYONECANPAY} & Current only & Per other flags & Crowdfunding \\
\texttt{SIGHASH\_ANYPREVOUT} & None (pubkey only) & All & Eltoo state replacement \\
\bottomrule
\end{tabular}
\end{table}
\end{definition}

\subsection{Notation Conventions}

This paper uses the following notation conventions:

\begin{table}[htbp]
\centering
\small
\begin{tabular}{@{}ll@{}}
\toprule
\textbf{Symbol} & \textbf{Meaning} \\
\midrule
$\calU$ & UTXO set \\
$U_{fund}$ & Fund UTXO (funding anchor) \\
$U_{state}^{(n)}$ & State UTXO with sequence number $n$ \\
$\tau$ & Transaction \\
$\delta$ & State transition function \\
$\RefOp(\cdot)$ & Read-only reference operation \\
$\Spend(\cdot)$ & Spend operation \\
$\prec$ & Partial order relation \\
$\cong$ & Isomorphism relation \\
$\perp$ & Orthogonality/Independence \\
\bottomrule
\end{tabular}
\end{table}
