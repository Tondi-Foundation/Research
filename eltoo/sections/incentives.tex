\section{Market Design and Incentive Mechanisms}

This architecture follows the \textbf{Minimal Intervention Principle}: the protocol defines the rules, while fees are determined by market competition. Fees serve as signal carriers for liquidity distribution.

\subsection{CSP Fee Structure}

\begin{definition}[Service Fee Model]
A CSP's revenue function is defined as:
$$ \Rcsp = \sum_{s \in \mathcal{S}} f_s \cdot V_s $$
where $f_s$ is the fee rate and $V_s$ is the transaction volume for service $s$.
\end{definition}

\begin{table}[htbp]
\centering
\caption{CSP Fee Schedule Structure}
\label{tab:fee_structure}
\small
\begin{tabularx}{\linewidth}{@{}l l X@{}}
\toprule
\textbf{Service} & \textbf{Fee Model} & \textbf{Economic Rationale} \\
\midrule
Channel Open & Fixed + 0.01\% & Overhead allocation \\
Routing & 0.1\% of value & Marginal cost pricing \\
JIT Liquidity & 0.5\% per use & Capital rental for instant inbound \\
Swap & 0.3--1.0\% & Market risk premium \\
Mixing & 0.1\% & Anonymity premium \\
\bottomrule
\end{tabularx}
\end{table}

\subsection{Liquidity Provider Economics}

\begin{definition}[LP Utility Function]
$$ \Ulp = r_{\mathrm{APY}} \cdot V_{\mathrm{dep}} - \rho \cdot \sigma^2_{\mathrm{slip}} - C_{\mathrm{opp}} $$
where:
\begin{itemize}
    \item $r_{\mathrm{APY}}$: Annualized yield from routing fees
    \item $V_{\mathrm{dep}}$: Deposited capital
    \item $\sigma^2_{\mathrm{slip}}$: Variance of slippage losses due to imbalanced flows
    \item $\rho \in [0.5, 2.0]$: Risk aversion coefficient
    \item $C_{\mathrm{opp}}$: Opportunity cost (e.g., DeFi staking yields)
\end{itemize}
\end{definition}

\begin{theorem}[Competitive Equilibrium]
In a market with $N \ge 3$ CSPs and free entry, fees converge to marginal cost over time:
$$ \lim_{\text{rounds} \to \infty} \Fee_{\mathrm{CSP}_i} \to \Cmarg + \epsilon $$
where $\epsilon > 0$ is an arbitrarily small profit margin.
\end{theorem}
\begin{proof}
If $\Fee > \Cmarg + \epsilon$, arbitrageurs enter at $\Fee' = \Fee - \delta$, capturing market share. This forces incumbents to lower prices, converging to marginal cost.
\end{proof}

\subsection{Anti-Collusion: L1 Fallback}

\begin{theorem}[Fee Upper Bound]
CSP fees are capped by the Layer 1 fallback cost:
$$ \Fee_{\mathrm{CSP}} \le \CLone + P_{\mathrm{privacy}} $$
\end{theorem}
This creates a \textbf{credible threat}: if $\Fee_{\mathrm{cartel}}$ exceeds this bound, users exit to L1 via the ``Right to Exit'' mechanism, making collusion unsustainable.

\subsection{Dynamic Fee Adjustment}

To manage congestion, we implement a multi-stage pricing curve.

\begin{figure}[htbp]
\centering
\begin{tikzpicture}[
    scale=0.9,
    font=\sffamily\scriptsize,
    >=Stealth
]
    % Axes
    \draw[->] (0,0) -- (8,0) node[right] {Utilization $U$};
    \draw[->] (0,0) -- (0,4.5) node[above] {Fee Multiplier};
    
    % Grid lines
    \draw[gray!30] (0,1) -- (8,1);
    \draw[gray!30] (0,2) -- (8,2);
    \draw[gray!30] (0,3) -- (8,3);
    \draw[gray!30] (0,4) -- (8,4);
    
    % X-axis labels
    \node[below] at (0,0) {0};
    \node[below] at (4.9,0) {0.7};
    \node[below] at (6.3,0) {0.9};
    \node[below] at (7,0) {1.0};
    
    % Y-axis labels
    \node[left] at (0,1) {1x};
    \node[left] at (0,2) {2x};
    \node[left] at (0,4) {4x};
    
    % Pricing curve
    \draw[blue, very thick] (0,1) -- (4.9,1);  % Base: 0-0.7
    \draw[blue, very thick] (4.9,1) -- (6.3,2);  % Linear: 0.7-0.9
    \draw[blue, very thick] (6.3,2) -- (7,4);  % Exponential: 0.9-1.0
    
    % Critical zone markers
    \draw[red, dashed] (4.9,0) -- (4.9,4.2);
    \draw[red, dashed] (6.3,0) -- (6.3,4.2);
    
    % Zone labels
    \node[above, font=\tiny] at (2.5,1.1) {Base Fee};
    \node[above, font=\tiny] at (5.6,1.4) {Linear};
    \node[right, font=\tiny] at (6.5,3) {Exp.};
    
    % Threshold annotations
    \node[below, font=\tiny, red] at (4.9,-0.3) {Surge};
    \node[below, font=\tiny, red] at (6.3,-0.3) {Congest.};
\end{tikzpicture}
\caption{Congestion Pricing Curve. Fees remain flat until 70\% utilization, then rise linearly, and finally exponentially to prevent resource exhaustion.}
\label{fig:congestion_curve}
\end{figure}

\begin{lstlisting}[caption={Dynamic Fee Calculation Logic}, label={lst:dynamic_fee}]
pub fn compute_dynamic_fee(utilization: f64) -> Fee {
    let base_fee = 100; // sompi
    let multiplier = if utilization > 0.9 {
        2.0 + (utilization - 0.9) * 20.0  // Exponential: 2x at 0.9, 4x at 1.0
    } else if utilization > 0.7 {
        1.0 + (utilization - 0.7) * 5.0   // Linear: 1x at 0.7, 2x at 0.9
    } else {
        1.0  // Base: flat 1x below 70%
    };
    Fee::new((base_fee as f64 * multiplier) as u64)
}
\end{lstlisting}

\subsection{Incentive Compatibility}

\begin{theorem}[Dominant Strategy]
Honest behavior is the dominant strategy for CSPs.
\end{theorem}

\begin{table}[htbp]
\centering
\caption{CSP Strategy Payoff Matrix}
\label{tab:payoff}
\small
\begin{tabularx}{\linewidth}{@{}l l X@{}}
\toprule
\textbf{Strategy} & \textbf{Net Benefit} & \textbf{Outcome Analysis} \\
\midrule
Honest & \textcolor{green!60!black}{\textbf{Positive}} & Earns fees + Reputation growth. \\
Delay & \textcolor{red}{\textbf{Negative}} & User churn $>$ Time value of locked funds. \\
Steal & \textcolor{red}{\textbf{Very Negative}} & Impossible (PTLC) + Slashing/Ban. \\
\bottomrule
\end{tabularx}
\end{table}

\begin{proof}
Let $S = \{\text{Honest}, \text{Delay}, \text{Steal}\}$.
Since PTLCs cryptographically prevent theft ($\Pr[\text{Success}|\text{Steal}] = 0$) and the L1 fallback option bounds the ``Delay'' utility ($U_{\text{delay}} < \text{ReputationCost}$), we have $U_{\text{honest}} > U_{\text{delay}} > U_{\text{steal}}$. Thus, Honest is the Nash Equilibrium.
\end{proof}

\subsection{Summary}

This mechanism achieves:
\begin{enumerate}
    \item \textbf{Competitive Pricing}: $\Fee \to \Cmarg$.
    \item \textbf{User Sovereignty}: Guaranteed by L1 fallback.
    \item \textbf{Dynamic Efficiency}: Prices reflect real-time scarcity via the congestion curve.
\end{enumerate}
