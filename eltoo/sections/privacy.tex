\section{Privacy and Anonymity Framework}

Traditional blockchain transparency exposes transaction graphs. This architecture implements \textbf{Selective Disclosure}, allowing users to autonomously control information scope.

\subsection{Threat Model and Anonymity Set}

\begin{definition}[Anonymity Set]
For a payment $p$ routed through $n$ CSP hops $\mathcal{H} = \{h_1, \dots, h_n\}$, the anonymity set size is the product of each hop's indistinguishable channel count:
$$ |\mathcal{AS}(p)| = \prod_{i=1}^{n} |\mathrm{Channels}_{h_i}| $$
For example, routing through 3 CSPs with 100 channels each yields $|\mathcal{AS}| = 100^3 = 10^6$. Payment $p$ is \textbf{$k$-anonymous} iff $|\mathcal{AS}(p)| \geq k$.
\end{definition}

\begin{table}[htbp]
\centering
\caption{Threat Model Classification}
\label{tab:threat_model}
\small
\begin{tabularx}{\linewidth}{@{}lXl@{}}
\toprule
\textbf{Adversary} & \textbf{Capability} & \textbf{Defense} \\
\midrule
Passive L1 & Graph Analysis & Mixing + Stealth Addr. \\
Active CSP & Timing Analysis & Dummy Traffic \\
Global & IP Correlation & Tor / I2P Integration \\
Quantum & ECDLP Attacks & Post-Quantum (Future) \\
\bottomrule
\end{tabularx}
\end{table}

\subsection{Payment Layer Privacy Analysis}

\subsubsection{PTLC vs.\ HTLC}

\begin{theorem}[PTLC Path Unlinkability]
Under the PTLC protocol, the probability of linking hops $(i, j)$ is negligible:
$$ \forall i \neq j: \Pr[\Link(\mathrm{hop}_i, \mathrm{hop}_j)] \le \epsilon_{\mathrm{negl}} $$
\end{theorem}

\begin{proof}
Each hop uses an independent scalar $r_i \in \mathbb{Z}_q$. An observer sees point locks $Q_i = r_i \cdot G$. Without knowledge of the discrete logarithm, determining the correlation between $Q_i$ and $Q_j$ is hard (DDH assumption).
\end{proof}

\begin{table}[htbp]
\centering
\caption{Privacy Comparison: PTLC vs.\ HTLC}
\label{tab:ptlc_vs_htlc}
\small
\begin{tabularx}{\linewidth}{@{}lXX@{}}
\toprule
\textbf{Feature} & \textbf{HTLC (Legacy)} & \textbf{PTLC (Proposed)} \\
\midrule
Linkability & \textbf{High} (Same Preimage) & \textbf{None} (Blind Scalar) \\
Amt.\ Hiding & Plaintext & Plaintext \\
Route Disc. & Exposed & Blinded \\
Math Basis & Hash Function & ECC Homomorphism \\
\bottomrule
\end{tabularx}
\end{table}

\subsection{Network Layer Privacy: Onion Routing}

Even with payment unlinkability, IP metadata remains a risk. We utilize the \textbf{SPHINX-Lite} protocol.

\subsubsection{Onion Packet Structure}
The packet is constructed recursively:
$$ P_{\mathrm{onion}} = \Enc_{pk_1}(r_1, \Enc_{pk_2}(r_2, \dots, \Enc_{pk_n}(r_n, m)\dots)) $$

\begin{figure}[htbp]
\centering
\begin{tikzpicture}[
    scale=0.85, transform shape,
    font=\sffamily\scriptsize,
    layer/.style={rectangle, draw, rounded corners, minimum height=0.6cm, align=center},
    >=Stealth
]
    % Layers (nested boxes)
    \node[layer, fill=blue!30, minimum width=6cm, minimum height=3cm] (l1) at (0,0) {};
    \node[above] at (l1.north) {$\Enc_{pk_1}(r_1, \dots)$};
    
    \node[layer, fill=blue!20, minimum width=4.5cm, minimum height=2cm] (l2) at (0,-0.2) {};
    \node[above] at (l2.north) {$\Enc_{pk_2}(r_2, \dots)$};
    
    \node[layer, fill=blue!10, minimum width=3cm, minimum height=1cm] (l3) at (0,-0.4) {};
    \node[above] at (l3.north) {$\Enc_{pk_3}(r_3, \dots)$};
    
    \node[layer, fill=green!20, minimum width=1.5cm, minimum height=0.5cm] (payload) at (0,-0.5) {Payload $m$};
    
    % Annotations
    \node[right=0.3cm of l1.east, align=left, font=\tiny] {Hop 1 Peels};
    \node[right=0.3cm of l2.east, align=left, font=\tiny] {Hop 2 Peels};
    \node[right=0.3cm of l3.east, align=left, font=\tiny] {Hop 3 Delivers};
    
    \draw[->, dashed] (l1.east) ++(0.1,0) -- ++(0.2,0);
    \draw[->, dashed] (l2.east) ++(0.1,0) -- ++(0.2,0);
    \draw[->, dashed] (l3.east) ++(0.1,0) -- ++(0.2,0);

\end{tikzpicture}
\caption{SPHINX-Lite Onion Structure. Each hop ``peels'' one layer of encryption, revealing only the next hop's routing info $(r_i)$, ensuring forward secrecy.}
\label{fig:onion_routing}
\end{figure}

\textbf{Key Properties:}
\begin{itemize}
    \item \textbf{Forward Secrecy}: Ephemeral keys per hop.
    \item \textbf{Bitwise Unlinkability}: Packet size remains constant at every hop via padding, preventing length analysis.
\end{itemize}

\subsection{Privacy-Performance Tradeoff}

\begin{theorem}[Privacy Cost]
Privacy enhancement incurs latency overhead that scales with the routing path length $n$ and network-layer protection:
$$ T_{\mathrm{latency}} = T_{\mathrm{base}} + n \cdot T_{\mathrm{hop}} + T_{\mathrm{overlay}} $$
where $T_{\mathrm{base}} \approx 100$ms is the direct payment latency, $T_{\mathrm{hop}} \approx 50$ms is the per-hop PTLC overhead, and $T_{\mathrm{overlay}}$ is the network anonymization cost (0 for clearnet, $\sim$3s for Tor).
\end{theorem}

\begin{table}[htbp]
\centering
\caption{Privacy Mode Tradeoffs}
\label{tab:privacy_modes}
\small
\begin{tabularx}{\linewidth}{@{}lXll@{}}
\toprule
\textbf{Mode} & \textbf{Description} & \textbf{Latency} & \textbf{Anonymity Set} \\
\midrule
Direct & Single-hop, no mixing & $\sim$100ms & $1$ (None) \\
Single CSP & 1-hop through CSP ($\sim$100 channels) & $\sim$150ms & $\sim 10^2$ \\
Multi CSP & 3-hop through CSPs & $\sim$250ms & $\sim 10^6$ \\
Tor + Multi & 3-hop + Tor overlay & $\sim$3.5s & $>10^8$ \\
\bottomrule
\end{tabularx}
\end{table}

\subsection{Stealth Addresses}

To protect receiver identity ($A, B$), sender generates a one-time destination $P_{\mathrm{stealth}}$:
\begin{equation}
P_{\mathrm{stealth}} = H(r \cdot B) \cdot G + A
\end{equation}
where $r$ is a random nonce. Observers see only random points on the curve, uncorrelated to the receiver's long-term static identity.

\subsection{Summary}

The architecture provides a spectrum of privacy defenses:
\begin{enumerate}
    \item \textbf{Payment}: PTLC Unlinkability.
    \item \textbf{Network}: Onion Routing (IP Hiding).
    \item \textbf{Identity}: Stealth Addresses.
    \item \textbf{Balance}: Confidential Transactions (Pedersen).
\end{enumerate}
